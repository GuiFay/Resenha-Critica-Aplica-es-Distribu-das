\documentclass[12pt, a4paper]{article}
\usepackage[utf8]{inputenc}
\usepackage[brazilian]{babel}
\usepackage[T1]{fontenc}
\usepackage{graphicx}
\usepackage{geometry}
\usepackage{color,soul}
\DeclareRobustCommand{\hlcyan}[1]{{\sethlcolor{cyan}\hl{#1}}}
\usepackage[document]{ragged2e}

\newcommand{\+}[1]{\ensuremath{\mathbf{#1}}} % comando para escrever matrizes em negrito \+A= A

\geometry{top=2.5cm, bottom=2.5cm, right=2.5cm, left=2.5cm}

\begin{document}
\noindent Universidade de Brasília UnB \hfill ENE394742 - Aplicações Distribuídas\\

\noindent PPGEE - Programa de Pós-Graduação em Engenharia Elétrica \hfill 2020\\ % Inserir o nome do programa que está participando.
\noindent Prof. Rafael Timoteo de Sousa Junior \\
\noindent Aluno: Nome completo do aluno \\ % Inserir o nome completo do aluno
\center{Resenha do artigo:}
\center{\large{Título do artigo}} % Inserir o título do artigo
\center{\textit{Autores do artigo}} % Inserir o nome dos autores dos artigos
\justify

%%%%%%% Conteúdo da resenha %%%%%%%%%%%

\begin{itemize}
    \item Apresentar a obra: situar o leitor descrevendo em poucas linhas todo o conteúdo do texto resenhado
    \item Descrever a estrutura: falar sobre a divisão em capítulos, em seções, sobre o foco narrativo ou até, de forma sutil, o número de páginas do texto completo
    \item Descrever o conteúdo: utilizar de 3 a 5 parágrafos para resumir claramente o texto resenhado
    \item Analisar de forma crítica: nessa parte, e apenas nessa parte, dar uma opinião. Argumentar baseando-se em teorias de outros autores, fazendo comparações ou até mesmo utilizando-se de explicações que foram dadas em aula. É difícil encontrar resenhas que utilizam mais de 3 parágrafos para isso, porém não há um limite estabelecido, podendo haver uma extensão criteriosa em função do tamanho do conteúdo resenhado
    \item Recomendar a obra: analisar para quem o texto realmente é útil (se for útil para alguém), utilizando elementos sociais ou pedagógicos, como idade, escolaridade, renda etc.
\end{itemize}







\end{document}